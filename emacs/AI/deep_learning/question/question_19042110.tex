方博士反向传播算法面试题精选

1. 请写出全微分公式

2. 梯度等于偏导数乘以学习步长的结果的相反数,请问这个公式是怎么推导出来的?为什么其中没有deltaY?
3. 要想让上述梯度公式成立,损失函数必须满足什么条件?为什么?
4. 请问全微分公式针对的是多元函数还是复合函数或者两者都有?为什么
5. 请问链式法则针对的是多元函数还是复合函数或者两者都有?为什么
多元函数
6. 请问批量梯度下降法、随机梯度下降法和小批量梯度下降法有什么区别?在TF中如何实现上述三种方法?

7. 损失函数和目标函数有什么区别?
目标函数是你的应用场景所要达到的目标
目标函数为y = ....

损失函数来源于目标函数,它是根据目标函数计算出来的.
损失函数为:loss = ...

8. 设一个NN网络中有一个relu激活层形如y=relu(x),其中x是输入向量,含4个神经元。y是输出层。假设在一次前向传播中已经知道x=[1.2, -0.3, 0.5, -1.7]。在相应的反向传播计算中,已知y向量中的四个神经元的梯度分别是[0.6,-0.4,1.1,1.5]。请问x的四个神经元的梯度分别是多少?

9. 设有网络y=f(x, w),x是输入,y是输出,w是f中的参数(可训练变量)。当w达到最优时下面那些结论是正确的:A. f对w的偏导数等于0; B: f对x的偏导数等于0; C: y等于0;D:梯度恒等于0

方博士CNN面试题精选

1. 判断对错:在卷积操作中,如果图片的格式不同(比如NCHW和NWHC格式),则产生的可训练参数数量也不同。
错,发生的参数与图片的参数没有关系
2. 判断对错:TF中Saver可用来保存和恢复当前模型的参数,但它只保存可训练参数,不保存不可训练参数。
错

卷积操作 k * k * in * out + out
什么跟长宽有着呢?
计算量
flops = ...(这个和长宽有关)
3. 判断对错:转置卷积是卷积的逆操作

4. 图像A的大小是64*64*17,经过一个核为5*5的卷积之后变成图像B,大小为60*60*31。请问:a)这样的卷积操作有可能吗?b)这个卷积操作的padding和strides各等于多少?c)B中的元素B[13][21][22]sh受到A中哪些元素的影响?

5. 卷积操作是线性的还是非线性的?
线性
6. 在内存足够的情况下,是不是卷积核越大越好?
7. 假设一次卷积操作中,原图片大小为h*w*c1,输出图片大小为h*w*c2,核的大小为k*k,问这次卷积操作共进行了多少次浮点乘法运算(不包括偏置)?

8. 为了把彩色图变成黑白图,可以把每个像素点的R、G、B的值分别乘以一个固定的系数再求和。请问这个操作相当于拼接,卷积,反卷积,还是池化?
9. 对一个200*200*3的图片进行Batch Normalize操作,会产生多少平均值和标准差?
10. 以下关于Batch Normalize操作的说法哪些是正确的?a) axis不同,每个点的平均值和标准差就会不同;b) axis不同,伽马和贝塔参数的数量就不同;c) axis不同,BN操作的结果是不一样的。
11. 以下关于Batch Normalize操作的说法哪些是正确的?a) BN操作计算出了精确的平均数和标准差;b) BN操作中产生的动量可以被反向传播优化;c) BN操作中产生的所有gamma和beta参数都可以被反向传播优化;d) 以上说法都不正确
12. 在预测(inference)时,BN操作的参数training必须设为False,为什么?如果设为True会怎样?
13. 设有一个4分类问题,每个分类是互斥的。4个分类分别是A、B、C、D。设有一个A类样本a,经过模型的前向传播计算后得到预测概率是[0.1, 0.5, 0.2, 0.2]。则a的交叉熵等于多少?所有概率之和应该等于1这句话对吗?
14. 设有一个4分类问题,每个分类不是互斥的。4个分类分别是A、B、C、D。设有一个样本s的类别是B和D,经过模型的前向传播计算后得到预测概率是[0.1, 0.5, 0.2, 0.2]。则s的交叉熵等于多少?所有概率之和应该等于1这句话对吗?
15. 以下关于TF中Saver的说法哪些是正确的:a) Saver保存模型的参数的同时也保存了张量和图(Graph);b) 在一个图下保存的模型是不能够在另一个图下恢复;c) 如果张量的名字改变就不能回复一个模型;d) 以上说法都不正确
16. 写出用L2_Normalize对序列[2, 0, 4]进行正则化的结果;
17. 写出用L1_Normalize对序列[2, 0, 4]进行正则化的结果; 
18. 写出对序列[2, 0, 4]进行正态分布标准化的结果;


方博士高等数学面试精选:
1. 请给出当x趋近a时, f(x)趋近于b的数学定义
2. 请给出当x趋近+∞时, f(x)趋近于b的数学定义
3. 求a**x的导数
4. 请写出泰勒公式
5. 请问什么是中值定理和洛比塔法则?