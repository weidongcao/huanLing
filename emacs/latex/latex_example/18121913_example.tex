\documentclass{article}

\begin{document}
\title{This Is A \LaTeX Demo}
\author{Weidong Cao}
\date{2018/12/19}
\maketitle
\begin{abstract}
  In this article, I shall discuss some of the fundamental topics in
  producing s structured document. This document itself does not go into
  much depth,but is instead the output of an example of how to implement
  这是我的第一个LaTeX论文
\end{abstract}

\section{Introduction}

This small document is designed to illustrate how easy it is to create a well structured
document within \LaTeX\citee{lamport94}. You should quickly be able to see how the article
looks very professional,despite the content being far from acacemic. Titles, section

\section{Structure}

One of the great advantages of \LaTeX{} is that all it needs to know is
the struceure of a document, and then it will take care of the layout

and presentation itself. So, here we shall begin looking at how exactly

you tell \LaTeX{} what it needs to know about your document.

\subsection{Top Matter}

\label{sec:top-matter}

The first thing you normally have is a title of the document, as well as

information about the author and date of publication. In \LaTeX{} terms

this is all generally referred to as \emph{top matter}.

\subsection{Article Information}

\begin{itemize}
\item \verb|\title{}| --- The title of the article
\item \verb|\date| --- The date. use:
  \begin{itemize}
  \item \verb|\date{\today}| --- to get the date that the document is typeset.
  \item \verb|\date{}| --- for no date.
  \end{itemize}

\end{itemize}

\subsection{Author Information}
\label{sec:info}

The basic article class only provides the one command:
\begin{itemize}
\item \verb|\author{}| --- The author of the document.
\end{itemize}

It is common to not only include the author name,but to insert new lines(\verb|\\|)

after and add things such as address and email details. For a slightly more logical
approach, use the AMS article class (\emph{amsart}) and you have the following extra
commands:

\begin{itemize}
\item \verb|address| --- The author's address. Use the new line command (\verb|\\|) for line\item \verb breaks
\item \verb|thanks| --- Where you put any acknowledgments.
\item \verb|eamil| --- The author's email address.
\item \verb|urladdr| --- The URL for the author's web page.
\end{itemize}

\subsection{Sectioning Commands}
\label{sec:info}

The commands for inserting sections are fairly intuitive. of course,
certain commands are appropriate to different document classes.
for example, a book has chapters but a article does't.

% a Simple table. The Center environment is first setup, otherwise the
% table is left aligned. The tabular environment is what tells LaXeX
% thatttt the data within is data for the table.

\begin{center}
  \begin{tabular}{|l|l|}
% The tabular environment is what tells Latex that the data within is
% data for the table. The arguments say that there will be two
% columns, both left justified (indicated by the 'l', you could also)
% \item have 'c' or 'r'. The bars '|' indicate vertical lines throughout
% \item the table.
    \hline
    Command & Level \\ \hline
    \verb|\part{}| & -1 \\
    \verb|\chapter{}| & 0 \\
    \verb|\section{}| & 1 \\
    \verb|\subsection{}| & 2 \\
    \verb|\subsubsection{} | & 3 \\
    \verb|\paragraph{} | & 4 \\
    \verb|\subparagraph{} | & 5 \\
    \hline
  \end{tabular}
\end{center}
\begin{thebibliography}{99}
\bibitem{lamport94}
  Leslie Lamport,
  \emph{\LaTeX: A Document Preparation System}.
  Addison Wesley, Massachusetts,
  2nd Edition,
  1994.
\bibitem{wikibooks}
  http://en.wikibooks.org/wiki/LaTeX/example.tex
\end{thebibliography}
\end{document}
